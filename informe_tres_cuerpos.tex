
\documentclass[12pt,a4paper]{article}
\usepackage[utf8]{inputenc}
\usepackage[spanish]{babel}
\usepackage{amsmath, amssymb}
\usepackage{graphicx}
\usepackage{hyperref}
\usepackage{geometry}
\geometry{margin=2.5cm}

\title{Simulación Numérica del Problema de Tres Cuerpos: Sol, Tierra y Venus}
\author{}
\date{\today}

\begin{document}

\maketitle

\section*{Resumen}
Se implementó una simulación numérica del problema de tres cuerpos usando Python y Matplotlib para modelar las interacciones gravitacionales entre el Sol, la Tierra y Venus. El código resuelve las ecuaciones de movimiento mediante un integrador explícito (actualización de velocidades y luego posiciones — Euler semi-implícito), registra trayectorias y genera una animación interactiva con control de velocidad en tiempo real. Los resultados muestran órbitas estables y la relación esperada entre distancias orbitales y periodos (Venus completa más órbitas que la Tierra). Se discuten limitaciones numéricas y mejoras propuestas (RK4, suavizado, control de energía).

\section{Introducción}
El problema de los tres cuerpos es un problema clásico de la mecánica celeste: dado tres masas que interactúan únicamente por gravedad newtoniana, predecir sus trayectorias. A diferencia del caso de dos cuerpos, no existe una solución analítica general y las soluciones pueden presentar comportamiento caótico. El objetivo de esta práctica fue implementar una simulación numérica simple que permita:
\begin{itemize}
    \item Visualizar las trayectorias en 2D.
    \item Permitir control en tiempo real de la velocidad de simulación.
    \item Analizar la concordancia con predicciones teóricas básicas (leyes de Kepler y comportamiento orbital).
\end{itemize}

\section{Metodología}
\subsection{Ecuaciones físicas}
Se utiliza la ley de gravitación universal de Newton y la segunda ley de Newton. Para cada cuerpo $i$, la aceleración viene dada por:
\begin{equation}
    \vec{a}_i = G \sum_{j \neq i} \frac{m_j (\vec{r}_j - \vec{r}_i)}{\|\vec{r}_j - \vec{r}_i\|^3}
\end{equation}
donde $G = 6.67430 \times 10^{-11} \, \mathrm{m^3\,kg^{-1}\,s^{-2}}$.

\subsection{Integrador numérico}
En cada paso de tiempo $\Delta t$ (variable según \texttt{velocidad\_factor}) se realiza:
\begin{enumerate}
    \item Calcular aceleraciones $\vec{a}_i$ por la suma sobre todos los pares.
    \item Actualizar velocidades: $\vec{v}_i \leftarrow \vec{v}_i + \vec{a}_i \, \Delta t$.
    \item Actualizar posiciones: $\vec{r}_i \leftarrow \vec{r}_i + \vec{v}_i \, \Delta t$.
\end{enumerate}
Este esquema (actualización de velocidades antes que posiciones) es el método de Euler semi-implícito (Euler-Cromer), más estable para sistemas oscilatorios que el Euler explícito.

\subsection{Condiciones iniciales}
Se usaron valores aproximados reales:
\begin{itemize}
    \item Sol: $m = 1.989 \times 10^{30}$ kg, posición $(0,0)$.
    \item Tierra: $m = 5.972 \times 10^{24}$ kg, posición $(1.496\times 10^{11}, 0)$ m, velocidad $(0, 29780)$ m/s.
    \item Venus: $m = 4.867 \times 10^{24}$ kg, posición $(1.082\times 10^{11}, 0)$ m, velocidad $(0, 35020)$ m/s.
\end{itemize}
Paso de tiempo base: $\Delta t_{base} = 3600$ s (1 hora), escalable con un factor máximo de $\times 64$.

\subsection{Implementación}
\begin{itemize}
    \item Lenguaje: Python 3.x
    \item Librerías: \texttt{numpy}, \texttt{matplotlib}
    \item Cálculo en tiempo real en función \texttt{update()} y almacenamiento de trayectorias.
\end{itemize}

\section{Resultados y observaciones}
\begin{itemize}
    \item Órbitas visibles: Tierra (azul) y Venus (naranja) describen órbitas casi circulares alrededor del Sol (amarillo).
    \item Venus completa más órbitas que la Tierra en el mismo intervalo, en concordancia con su menor distancia al Sol.
    \item El Sol muestra un ligero movimiento debido a la reacción gravitatoria de los planetas.
    \item Las órbitas se mantienen estables durante la simulación.
    \item Se observa una leve precesión en Venus debido a la interacción con la Tierra.
\end{itemize}

\section{Comparación con la teoría}
\begin{itemize}
    \item \textbf{Leyes de Kepler}: $T^2 \propto a^3$ se cumple cualitativamente.
    \item \textbf{Velocidad orbital}: consistente con $v \approx \sqrt{\frac{GM_{\odot}}{r}}$.
    \item \textbf{Conservación de momento}: el movimiento del Sol refleja la conservación del centro de masas.
\end{itemize}

\section{Limitaciones y desafíos}
\begin{itemize}
    \item El método de Euler-Cromer acumula error energético a largo plazo.
    \item $\Delta t$ grande introduce errores; pequeño aumenta el coste computacional.
    \item No se modelan colisiones ni se incluye \emph{softening}.
\end{itemize}

\section{Recomendaciones y mejoras}
\begin{itemize}
    \item Usar integrador RK4 o métodos \emph{symplectic}.
    \item Implementar \emph{softening} para evitar singularidades.
    \item Añadir cálculo y graficado de energía total y momento.
    \item Extender el sistema a más planetas y validar contra datos reales.
\end{itemize}

\section{Conclusiones}
La simulación reproduce de forma cualitativa el comportamiento esperado del sistema Sol–Tierra–Venus: órbitas estables, periodos relativos correctos y reacción del Sol al tirón gravitatorio. Es útil para enseñanza y visualización, aunque para análisis precisos se recomiendan integradores más avanzados y validaciones adicionales.

\section*{Referencias}
\begin{itemize}
    \item Newton, I. \emph{Philosophiæ Naturalis Principia Mathematica}.
    \item Murray C.D., Dermott S.F., \emph{Solar System Dynamics}. Cambridge Univ. Press.
    \item Datos astronómicos de tablas públicas.
\end{itemize}

\end{document}
